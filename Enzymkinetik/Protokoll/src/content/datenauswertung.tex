\newpage
\section{Datenauswertung}

\subsection{Konzentrations-Zeit-Diagramm}

Da die Absorptionen des Produktes mit Hilfe eines Photometers gemessen worden sind, wird die Lambert-Beer Gleichung umgeformt, um die einzelnen Konzentrationen zu erhalten. Dabei wird der Verdünnungsfaktor miteinbezogen. 

\begin{align}
  \begin{split}  
    A &= \epsilon \cdot c \cdot d \\
    c &= F \cdot \frac{A}{\epsilon \cdot d}
  \end{split}
\end{align}

\begin{table}[H]
  \begin{tabular}{ll}
    A...& Absorption \\
    $\epsilon$... & molarer Extinktionskoeffizient $[\epsilon = 1,88 \cdot 10^4 \frac{l}{mol \cdot cm}]$ \\
    d... & Schichtdicke $[1cm]$ \\
    c... & Konzentration $[\frac{nmol}{ml}]$ \\
    F... & Verdünnungsfaktor $[F = 6]$
  \end{tabular}
\end{table}

Dadurch, dass die Zeit und die Konzentrationen graphisch aufgetragen werden, werden die Reaktionsgeschwindigkeit der jeweiligen Substratkonzentration ermittelt.

\ncBild{ungehemmtKonzentrationZeit.png}{ungehmmtes Substrat}{ungehemmt}
\ncBild[0.9]{5mMKonzentrationZeit.png}{Substrat mit 5mM Inhibitorlösung}{5mM}
\ncBild[0.9]{50mMKonzentrationZeit.png}{Substrat mit 50mM Inhibitorlösung}{50mM}
\ncBild[0.9]{150mMKonzentrationZeit.png}{Substrat mit 150mM Inhibitorlösung}{150mM}

\subsection{Reaktionsgeschwindigkeiten}

Um die Anfangsgeschwindigkeiten(siehe Tabelle \ref{tab:Messwerte}) zu erhalten, ermittelt man die Steigungen der Zeit und Konzentration.  Für die Bestimmung der Reaktionsgeschwindigkeit wurde die Absorptionszunahme im Zeitraum 0-20 min für die Konzentrationen von 5 und 50 mM und 0-10 min für die Konzentrationen von ungehemmt und 150 mM verwendet. Da bei den Konzentrationen (5 und 50 mM) die Werte nicht nachvollziehbar sind, wurde von 0 auf 20 min die Steigung gemessen. Selbiges gilt für die ungehemmte und die 150 mM Reaktion, bei der die Steigung von 0 auf 20 min berechnet  wurde.

% Table generated by Excel2LaTeX from sheet 'Tabelle1'
\begin{table}[H]
  \centering
  \caption{Ungehemmt}
    \begin{tabular}{ccccc}
    \midrule
    $V_{pNPP} [\mu l]$ & $V_{Puffer} [ml]$ & $Abs_{t=0}$ & $Abs_{t=10}$ & $Abs_{t=20}$\\
    \midrule
    0     & 4,5   & 0     & 0     & 0 \\
    30    & 4,47  & 0,006 & 0,097 & 0,156 \\
    50    & 4,45  & 0,008 & 0,109 & 0,173 \\
    100   & 4,4   & 0,009 & 0,211 & 0,37 \\
    200   & 4,3   & 0,017 & 0,257 & 0,474 \\
    300   & 4,2   & 0,011 & 0,345 & 0,611 \\
    500   & 4     & 0,02  & 0,375 & 0,698 \\
    700   & 3,8   & 0,033 & 0,451 & 0,84 \\
    1000  & 3,5   & 0,026 & 0,434 & 0,868 \\
    1000  & 4     & 0,026 & 0,028 & 0,28 \\
    \bottomrule
    \multicolumn{5}{@{}p{0.6\textwidth}@{}}{\footnotesize Die letzte Lösung ($V_{pNPP} = 1000 \mu l$ und $V_{Puffer} = 4 ml$) wird gemacht, um zu schauen ob das Substrat nicht von sich aus zerfällt und wird in allen weiteren Berechnungen nicht mehr beachtet}\\
    \end{tabular}%
  \label{tab:addlabel}%
\end{table}%


Die Anfangsgeschwindigkeiten bilden eine Grundlage für das Michaelis-Menten Diagramm, Lineweaver-Burk Diagramm und den Dixon Plot. 


\subsection{Michaelis-Menten Diagramm}

Für das Michaelis-Menten Diagramm werden die Anfangsgeschwindigkeiten gegen die Substratkonzentration graphisch aufgetragen. Dadurch kann man die Maximalgeschwindigkeit vmax und die Michaelis-Menten Konstante Km ermitteln.
\\
Die Maximalgeschwindigkeit entspricht dem Wert der Geschwindigkeit der Asymptote. 
Um die Michaelis-Menten Konstante Km zu erhalten, wird die Maximalgeschwindigkeit halbiert und bei dem Wert die Substratkonzentration geschätzt. Sie entspricht der Michaelis-Menten Konstante. 

\begin{align}
  K_M = \frac{v_{max}}{2}
\end{align}

Maximalgeschwindigkeit als auch die Michaelis-Menten Konstante Km wurden händisch abgelesen (geschätzt).

\ncBild[0.8]{MichaelisMentenDiagramm.png}{Michaelis-Menten-Diagramm}{MichaelisMentenDiagramm}

Anmerkung: Entfernt wurden bei dem Diagramm folgende Werte, die als Ausreißer gelten: \\
\begin{center}
  \begin{tabular}{cc}
    100 $\mu l$ & \makecell{bei 50 mM kein Wert, da die Spitze \\ der Kolbenhuppipette ausgelaufen ist} \\
    & \\
    700 $\mu l$ & bei 150 mM Ausreißer
  \end{tabular}
  \subsection{Lineweaver-Burk Diagramm}
\end{center}

Das Lineweaver-Burk Diagramm ist eine andere Methode, um die Michaelis-Menten Konstante $K_M$ als auch die Maximalgeschwindigkeit $v_{max}$ zu ermitteln. Dabei werden die Anfangsgeschwindigkeit gegen Substratkonzentration jeweils reziprok ($ \frac{1}{v};\frac{1}{[S]} $)aufgetragen.
Der Schnittpunkt mit der y-Achse entspricht $\frac{1}{v_{max}}$ und der Schnittpunkt mit der x-Achse ist $\frac{-1}{K_M}$. Die Parameter wurden ebenfalls wie beim Michaelis-Menten Diagramm händisch geschätzt. 

Anmerkung: Die gleichen Ausreißer (siehe MM Diagramm) wurden auch in diesem Diagramm nicht berücksichtigt

\ncBild[0.8]{LineweaverBurkDiagramm.png}{Lineweaver-Burk Diagramm}{LineweaverBurkDiagramm}

In der folgenden Tabelle werden die Parameter beider Methoden gegenüber gestellt:

% Table generated by Excel2LaTeX from sheet 'Vergleichswerte vmax Km (MM LB)'
\begin{table}[H]
  \centering
  \caption{Michaelis-Menten vs Lineweaver-Burk}
    \begin{tabular}{lrrrr}
      & \\
          & \multicolumn{2}{c}{Michaelis-Menten} & \multicolumn{2}{c}{Lineweaver-Burk} \\
    \midrule
     & $v_{max}$ & $K_M$ & $v_{max}$ & $K_M$\\
    \midrule
    ungehemmt & 13,8  & 720   & 11,1  & 720 \\
    5mM   & 11,1  & 1400  & 10    & 720 \\
    50mM  & 6,8   & 1250  & 5,3   & 910 \\
    150mM & 4,8   & 1950  & 2,8   & 1000 \\
    \bottomrule
    \multicolumn{5}{@{}p{0.5\textwidth}@{}}{\footnotesize Vergleich der erhaltenen Werte $v_{max}$ und $K_M$ über die beiden Bestimmungsarten. Links sind die Werte die man mit der Michaelis-Menten Auftragung bekommt und rechts die, die man über das Lineweaver-Burk Diagramm erhält}
    \end{tabular}%
  \label{tab:addlabel}%
\end{table}%


\subsubsection{Ermittlung des Inhibitortyps}

Darüber hinaus gibt das Lineweaver-Burk Diagramm Auskunft über den Inhibitortyp des Inhibitors. 
In diesem Fall handelt es sich um eine Mischung aus kompetitivem und nicht-kompetitivem Inhibitor, da weder die Maximalgeschwindigkeit noch die Michaelis-Menten Konstante über die Konzentration konstant bleiben.

\subsection{Dixon-Plot, Bestimmung der Inhibitorkonstante}

Um die Inhibitorkonstante $K_I$ zu ermitteln, wird die reziproke Anfangsgeschwindigkeit gegen die Inhibitorkonzentration für die Substratkonzentrationen aufgetragen. Die Inhibitorkonzentration beim Schnittpunkt der Geraden entspricht der Inhibitorkonstante $K_I$ und diese wurde händisch abgeschätzt. 
Anmerkung: 30$\mu l$ und 50$\mu l$ Geraden wurden aufgrund großer Abweichungen entfernt.

\ncBild[0.8]{DixonPlot.png}{Dixon-Plot}{DixonPlot}

Da im Diagramm nicht ein exakter Schnittpunkt auftritt, liegt die Inhibitorkonstante in einem Interwall (200-300 $\frac{nmol}{ml}$).

Anmerkung: Verdünnung wurde berücksichtigt.

\subsection{Bestimmung der turnover number ($k_{cat}$), der katalytischen Effizienz ($k_1$) und der spezifischen Aktivität}

\subsubsection{turnover number ($k_{cat}$)}
Die Wechselzahl sagt die Anzahl von Formelumsätzen in einer gewissen Zeit in einer Katalyse, die ein Katalysator beschleunigen kann, aus.
Die turnover number wird ermittelt, indem man die Maximalgeschwindigkeit durch die Enzymkonzentration dividiert wird. Dabei wird die uninhibierte Reaktion aus dem Michaelis-Menten Diagramm für die Berechnung entnommen. 

%fehler?
\begin{align}
  \begin{split}
    k_{cat} &= \frac{v_{max}}{[E]}  = \frac{13,8 \, [\frac{nmol}{ml \cdot min}]}{1,8 \cdot 10^{-8} \, [\frac{mol}{l}]}\\
    k_{cat} &= 7,8 \cdot 10^2 \, min^{-1} 
  \end{split}
\end{align}

\begin{table}[H]
  \begin{tabular}{lll}
    $k_{cat}$...& turnover number $[min^{-1}]$ & \\
    $v_{max}$... & Maximalgeschwindigkeit $[\frac{nmol}{ml \cdot min}]$ & $v_{max} = 13,8 \frac{nmol}{ml \cdot min}$\\
    E... & Enzymkonzentration $[\frac{mol}{l}]$ & $[E] = 1,8 \cdot 10^{-8} \frac{mol}{l}$ \\
  \end{tabular}
\end{table}

\subsubsection{katalytische Effizienz}

Der $K_M$-Wert wird wie in 3.6.1 für die uninhibierte Reaktion genommen.
\begin{align}
  \begin{split}
    k_1 &= \frac{k_{cat}}{K_M} \\
    k_1 &= 5,6 \cdot 10^7 \frac{l}{mol \cdot min}
  \end{split}
\end{align}

\begin{table}[H]
  \begin{tabular}{lll}
    $k_{1}$...& katalytische Effizienz $[\frac{ml}{mol \cdot min}]$ & \\
    $K_M$... & Michaelis-Menten Konstante $\frac{nmol}{ml}$ \\
  \end{tabular}
\end{table}

\subsubsection{spezifische Aktivität}
Die Spezifische Aktivität gibt die Wirksamkeit eines Enzyms an beziehungsweise die Umsetzung der Menge eines Substrats in einem bestimmten Zeitraum. \\
Die Maximalgeschwindigkeit wird wie in 6.1 für die uninhibierte Reaktion genommen. 

\begin{align}
  \begin{split} 
    A_s &= \frac{v_{max}}{m_{Protein}} \\
    A_s &= 7,0 \frac{\mu mol}{mg \cdot min}
  \end{split}
\end{align}

\begin{table}[H]
  \begin{tabular}{lll}
    $m_{Protein}$...& Proteinmenge in Reaktionslösung $[0,02 \, \frac{mg}{ml}]$ & \\
    $A_s$... & spezifische Aktivität $\frac{\mu mol}{mg \cdot min}$ \\
  \end{tabular}
\end{table}
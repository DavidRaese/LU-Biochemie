\section{Einleitung}

Die Aufgabe dieses Moduls bestand darin, eine enzymkinetische Studie durchzuführen. Solche Studien geben über mehrere Parameter($K_M, K_{cat}, v_{max}$) Aufschluss, welche dann z.B. für die Suche nach neuen Arzneimitteln oder die mechanistische Aufklärung von Enzymen genutzt werden können. Bei enzymkatalysierten Reaktionen geht man davon aus, dass ein Enzym mit einem Substrat reagiert, dabei bildet sich ein Enzymsubstratkomplex, welcher dann das Substrat zum Produkt umsetzt. Diese Reaktionen ist in jedem Schritt reversibel. 
 \begin{align}
    \mathrm{E} + \mathrm{S} \xrightleftharpoons[k_{-1}]{k_1} \mathrm{ES}
    \xrightleftharpoons[k_{-2}]{k_2} \mathrm{E} + \mathrm{P}
 \end{align}
 Zu Vereinfachung wird die Reaktionsgeschwindigkeit nahe dem Zeitpunkt Null ($v_0$) betrachtet, da hier angenommen werden kann, dass die Rückreaktion von Enzym und Produkt zum Enzymsubstratkomplex noch nicht in nennenswertem Umfang stattfindet.  
 \begin{align}
    \mathrm{E} + \mathrm{S} \xrightleftharpoons[k_{-1}]{k_1} \mathrm{ES}
    \xrightleftharpoons{k_2} \mathrm{E} + \mathrm{P}
 \end{align}
 Daher wird im ersten Versuch $v_0$ für mehrere Substratkonzentrationen in der Anfangsphase, wenn sich noch kaum Produkt gebildet hat, gemessen. \\
 Dafür wird eine APase (Phosphatase) mit einen Modellsubstrat (\textit{para}Nitrophenylphosphat) zu einem bei 405nm photometrisch detektierbaren Produkt (paraNitrophenol) umgesetzt (Abbildung \ref{Abbildung1}).
 Bei drei weiteren Versuchen wird noch anionisches Phosphat zur Inhibierung mit verschiedenen Konzentrationen zugegeben. Durch diese Messungen kann mit dem Michaelis-Mentson Diagramm und weiteren Berechnungen dann auf die Geschwindigkeit $v$ der Reaktion, den $K_M$-Wert, welcher ein Maß für die Stabilität des Enzymsubstratkomplexes ist, den Inhibitortypen, die turnover number ($k_{cat}$) und weitere Parameter geschlossen werden.
  
 \ncBild{reaktion}{Diese Abbildung zeigt, welche Reaktionen hauptsächlich stattfinden}{Abbildung1}
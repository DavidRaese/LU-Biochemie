\section{Diskussion}

Bei der Auftragung der erhaltenen Anfangsgeschwindigkeiten wurden für die Reaktionen ungehemmt, und gehemmt 150 mM gute Regressionsgeraden erhalten. Bei den Konzentrationen der 50mM gehemmten und 5 mM gehemmten Reaktion sind teils starke Ausreißer vorhanden wodurch wir auf die Konzentrationszunahme von 0-20 min zurückgreifen mussten. Solche Fehler sind auf ungenaues Pipettieren oder unsachgemäße Probenvorbereitung zurückzuführen und könnten durch Mehrfachbestimmungen minimiert werden. 
Des Weiteren stimmen die Werte von $K_M$ und $v_{max}$ aus den zwei verschiedenen Bestimmungen (Michaelis-Menten und Lineweaver-Burk) nicht sehr gut überein. Ein Grund dafür ist, dass die Werte aus dem Michaelis-Menten Diagramm nur händisch geschätzt werden konnten und kleine Fehler im Lineweaver-Burk Diagramm durch die reziproke Auftragung verstärkt werden. Dies ist auch der Grund warum wir für weitere Berechnungen die Werte aus dem Michaelis-Menten-Diagramm verwendet haben. Das Lineweaver-Burk Diagramm wurde lediglich zur Bestimmung des Inhibitortyps verwendet (gemischter Inhibitor).
Die Inhibitorkonstante konnte ebenfalls nur geschätzt werden da sich kein genauer Schnittpunkt der Geraden ergab.
Es wurden verschieden Literaturwerte für $A_{spez}$ der APase aus Kartoffeln gefunden. Diese Liegen zwischen 0,5-10 units pro mg Feststoff wobei ein unit = 1 $\mu mol$ pNPP pro Minute bei pH=4,8 und 37$^\circ$C. Unsere spez. Aktivität liegt bei 0,7 $\frac{\mu mol}{mg \cdot min}$ \cite{Aspez}  in dieser Reichweite. Um genauer vergleichen zu können, müsste uns jedoch die genaue Enzymbezeichnung bekannt sein. 
